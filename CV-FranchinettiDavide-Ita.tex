%%%%%%%%%%%%%%%%%%%%%%%%%%%%%%%%%%%%%%%%%
% Plasmati Graduate CV
% LaTeX Template
% Version 1.0 (24/3/13)
%
% This template has been downloaded from:
% http://www.LaTeXTemplates.com
%
% Original author:
% Alessandro Plasmati (alessandro.plasmati@gmail.com)
%
% License:
% CC BY-NC-SA 3.0 (http://creativecommons.org/licenses/by-nc-sa/3.0/)
%
% Important note:
% This template needs to be compiled with XeLaTeX.
% The main document font is called Fontin and can be downloaded for free
% from here: http://www.exljbris.com/fontin.html
%
%%%%%%%%%%%%%%%%%%%%%%%%%%%%%%%%%%%%%%%%%

%----------------------------------------------------------------------------------------
%	PACKAGES AND OTHER DOCUMENT CONFIGURATIONS
%----------------------------------------------------------------------------------------

\documentclass[a4paper,10pt]{article} % Default font size and paper size

\usepackage[italian]{babel}

\usepackage{fontspec} % For loading fonts
\defaultfontfeatures{Mapping=tex-text}
\setmainfont[SmallCapsFont = Fontin-SmallCaps]{Fontin-Regular} % Main document font

\usepackage{xunicode,xltxtra,url,parskip} % Formatting packages

\usepackage[usenames,dvipsnames]{xcolor} % Required for specifying custom colors

\usepackage[big]{layaureo} % Margin formatting of the A4 page, an alternative to layaureo can be \usepackage{fullpage}
% To reduce the height of the top margin uncomment: \addtolength{\voffset}{-1.3cm}

\usepackage{hyperref} % Required for adding links	and customizing them
\definecolor{linkcolour}{rgb}{0,0.2,0.6} % Link color
\hypersetup{colorlinks,breaklinks,urlcolor=linkcolour,linkcolor=linkcolour} % Set link colors throughout the document

\usepackage{titlesec} % Used to customize the \section command
\titleformat{\section}{\Large\scshape\raggedright}{}{0em}{}[\titlerule] % Text formatting of sections
\titlespacing{\section}{0pt}{8pt}{8pt} % Spacing around sections

\begin{document}

\pagestyle{empty} % Removes page numbering

%----------------------------------------------------------------------------------------
%	NAME AND CONTACT INFORMATION
%----------------------------------------------------------------------------------------

\par{\centering{\Huge \textsc{Davide Franchinetti}}\bigskip\par} % Your name

\section{Informazioni Personali}

\begin{tabular}{rl}
\textsc{Luogo e Data di Nascita:} & Italia | 9 giugno 1995 \\
\textsc{Residenza:} & Strada Privata Merli 6, Novara, 28100, Italia \\
%\textsc{Phone:} & (Ask privately)\\
\textsc{Email:} & \href{mailto:davide.franchinetti@gmail.com}{\underline{davide.franchinetti@gmail.com}}\\
\textsc{GitHub:} & \href{https://github.com/rdxdkr}{\underline{rdxdkr}}\\
\textsc{StackOverflow:} & \href{https://stackoverflow.com/users/9212745/rdxdkr}{\underline{rdxdkr}}\\
\end{tabular}

%----------------------------------------------------------------------------------------
%	EDUCATION
%----------------------------------------------------------------------------------------

\section{Istruzione}

\begin{tabular}{rl}	
\textsc{apr 2022} & Laurea Triennale in Informatica\\
& \textbf{Università del Piemonte Orientale}, Vercelli\\
& \textsc{Tesi:} ``Controllo elettronico degli accessi durante una pandemia''\\
& \small \textsc{Relatore:} Prof. Marco Guazzone\\
& \normalsize \textsc{Voto:} 109/110\\
\end{tabular}

%----------------------------------------------------------------------------------------
%	WORK EXPERIENCE 
%----------------------------------------------------------------------------------------

\section{Esperienze Lavorative}

\begin{tabular}{r|p{10.5cm}}
\textsc{nov 2020 - apr 2021} & Tirocinante presso \textsc{MasterSoft S.R.L.}, Novara \emph{}\\
& \footnotesize{Ho sviluppato un sistema client-server che permette alle persone di eseguire accessi in sicurezza di persona o da remoto, sul quale è basata la Tesi Triennale. Gli accessi di persona comprendono il riconoscimento del viso e della mascherina, la scansione di codici a barre, la misurazione della temperatura e funzionalità di Text To Speech in un'applicazione Android. Valutazione "Eccellente" per il lavoro svolto, le capacità comunicative con gli altri membri del gruppo e la proattività nell'imparare e applicare nuovi concetti e tecnologie in modo autonomo.}\\
\multicolumn{2}{c}{}\\

%------------------------------------------------

\textsc{2013 - oggi} & Tecnico di Computer Amatoriale, \textsc{Lavoro Autonomo} \emph{}\\
& \footnotesize{Ho assemblato e aggiornato molti computer desktop per me e i miei parenti, dopo aver svolto abbastanza ricerche su ultime novità hardware, prezzi e tecnologie disponibili. Ho scritto guide, tutorial e documentazione rivolti a persone inesperte, per aiutarle a capire come utilizzare un certo sistema operativo (es: Linux) o software.}
\end{tabular}

%----------------------------------------------------------------------------------------
%	SCHOLARSHIPS AND ADDITIONAL INFO
%----------------------------------------------------------------------------------------

\section{Certificati}

\begin{tabular}{rp{10.5cm}}
\textsc{giu 2012} & First Certificate in English\\
& \textbf{University of Cambridge, ESOL Examinations}\\
& \textsc{Livello:} B2
\end{tabular}

%----------------------------------------------------------------------------------------
%	LANGUAGES
%----------------------------------------------------------------------------------------

\section{Lingue}

\begin{tabular}{rl}
\textsc{Inglese:} & fluente\\
\textsc{Italiano:} & madrelingua\\
\end{tabular}

%----------------------------------------------------------------------------------------
%	COMPUTER SKILLS 
%----------------------------------------------------------------------------------------

\section{Competenze Tecniche}

\begin{tabular}{rl}
\textsc{Intermedio:} & Kotlin \footnotesize(Android, CameraX, ML Kit, Ktor)\normalsize, Java \footnotesize(JUnit 4)\normalsize, C\\
\textsc{Base:} & Rust, Python \footnotesize(Django)\normalsize, C\# \footnotesize(motore grafico Unity, ASP.NET Core)\normalsize, PostgreSQL, {\sffamily\LaTeX}\setmainfont[SmallCapsFont=Fontin SmallCaps]{Fontin-Regular}\\

\end{tabular}

%----------------------------------------------------------------------------------------
%	INTERESTS AND ACTIVITIES
%----------------------------------------------------------------------------------------

\section{Interessi e Attività}

Camminare all'aperto, Calisthenics, Cucinare \footnotesize(primi e secondi piatti)\normalsize, Software Open Source\\
Stranezze e particolarità dei linguaggi di programmazione \footnotesize(sintassi e idiomi)\normalsize\\
Videogiochi, Musica \footnotesize(attuale artista preferito: \href{https://youtu.be/tsmThCBkBUo}{\underline{Al Di Meola}})\normalsize, \href{https://youtu.be/s0lyDViRbTY}{\underline{Rasatura Tradizionale}}

%----------------------------------------------------------------------------------------

\end{document}
